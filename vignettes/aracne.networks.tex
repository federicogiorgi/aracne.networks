\documentclass{article}
\usepackage{fullpage}
\usepackage{hyperref}
\usepackage{authblk}
%\VignetteIndexEntry{Using aracne.networks}

\title{aracne.networks, a data package containing gene regulatory networks assembled from TCGA data by the ARACNe algorithm}
\author[1,2]{Federico M. Giorgi}
\author[1,3]{Mariano J. Alvarez}
\author[1]{Andrea Califano}
\affil[1]{Department of Systems Biology, Columbia University, New York, USA}
\affil[2]{CRUK, Cambridge University, Cambridge, UK}
\affil[3]{DarwinHealth Inc., New York, USA}
\date{\today}

\usepackage{Sweave}
\begin{document}
\Sconcordance{concordance:aracne.networks.tex:aracne.networks.Rnw:%
1 15 1 1 0 4 1 1 4 11 1 1 2 1 0 1 1 10 0 1 2 5 1 1 2 1 0 1 1 16 0 1 2 %
12 1}

\maketitle




%-----------
\section{Overview of aracne.networks data package}\label{sec:overview}
The \emph{aracne.networks} data package provides context-specific transcriptional regulatory networks (also called interactomes or regulons) reverse engineered by the ARACNe algorithm from The Cancer Genome Atlas (TCGA) RNAseq expression profiles.

\paragraph{ARACNe networks}
This package contains 24 Mutual Information-based networks assembled by ARACNe-AP \cite{Giorgi2016} with default parameters (MI p-value = $10^{-8}$, 100 bootstraps and  permutation seed = 1).
ARACNe-AP was run on RNA-Seq datasets normalized using Variance-Stabilizing Transformation \cite{Anders2010}.
The raw data was downloaded on April 15\textsuperscript{th}, 2015 from the TCGA website \cite{Weinstein2013}.
We follow the TCGA naming convention (e.g. BRCA = Breast Carcinoma) to name the individual context-specific networks.

\begin{Schunk}
\begin{Sinput}
> library(aracne.networks)
> data(package="aracne.networks")$results[, "Item"]
\end{Sinput}
\begin{Soutput}
 [1] "regulonblca" "regulonbrca" "reguloncesc" "reguloncoad" "regulonesca"
 [6] "regulongbm"  "regulonhnsc" "regulonkirc" "regulonkirp" "regulonlaml"
[11] "regulonlihc" "regulonluad" "regulonlusc" "regulonov"   "regulonpaad"
[16] "regulonpcpg" "regulonprad" "regulonread" "regulonsarc" "regulonstad"
[21] "regulontgct" "regulonthca" "regulonthym" "regulonucec"
\end{Soutput}
\end{Schunk}

\paragraph{Write a network to file}
The package contains a function to print individual networks into a file.
Four columns will be printed: the Regulator id, the Target id, the Mode of Action (MoA, inferred by Spearman correlation analysis \cite{Alvarez2016}) that indicates the sign of the association between regulator and target gene and ranges betrween -1 and +1, the Likelihood (essentially an edge weight that indicates how strong the mutual information for an edge is when compared to the maximum observed MI in the network, it ranges between 0 and 1). Further details about the \emph{regulon} object as a model for transcriptional regulation are present in the manuscript \cite{Alvarez2016}.

In the following example, we print the first 10 interactions from the bladder carcinoma (blca) network. The network genes are identified by Entrez Gene ids.
\begin{Schunk}
\begin{Sinput}
> data(regulonblca)
> write.regulon(regulonblca, n = 10)
\end{Sinput}
\begin{Soutput}
Regulator	Target	MoA	likelihood
10002	2648	0.994689591270463	0.886774633189913
10002	677827	0.116175345640136	0.707841406455471
10002	80152	0.999770437015603	0.950286744281199
10002	284382	-0.0368424333564396	0.0419762049859333
10002	9866	0.972066598154448	0.442238853411591
10002	283422	-0.574084929385018	0.260828476620346
10002	221613	-0.0959242601820319	0.717904706549976
10002	348174	0.953943934091558	0.814491117578869
10002	373509	0.704691385719852	0.244337186726846
10002	8803	-0.959165656086931	0.831653033754096
\end{Soutput}
\end{Schunk}

%-----------
\begin{thebibliography}{00}
\bibitem{Giorgi2016} Giorgi,F.M. et al. (2016) ARACNe-AP: Gene Network Reverse Engineering through Adaptive Partitioning inference of Mutual Information. Bioinformatics doi: 10.1093/bioinformatics/btw216.
\bibitem{Anders2010} Anders, S and Huber W. (2010) Differential expression analysis for sequence count data. Genome Biol 2010;11(10):R106
\bibitem{Weinstein2013} Weinstein J.N. et al. (2013) The cancer genome atlas pan-cancer analysis project. Nature Genetics 45, 1113-1120 2013
\bibitem{Alvarez2016} Alvarez M.J. et al. (2016) Functional characterization of somatic mutations in cancer using network-based inference of protein activity. Nature Genetics \emph{in press}.
\end{thebibliography}

%----------
\end{document}


